\documentclass[conference]{IEEEtran}
\begin{document}

\title{ARchitect: Construction Based AR*\\
{\footnotesize \textsuperscript{*}Note: Sub-titles are not captured in Xplore and
should not be used}
\thanks{Identify applicable funding agency here. If none, delete this.}
}

\author{\IEEEauthorblockN{1\textsuperscript{st} L. Sai Kiran Reddy}
\IEEEauthorblockA{\textit{Dept. of Comp. Sc. Engg} \\
\textit{SRMIST}\\
Chennai, India \\
saikiranreddy1001@gmail.com}
\and
\IEEEauthorblockN{2\textsuperscript{nd} Vijay Krishna Vallabhaneni}
\IEEEauthorblockA{\textit{Dept. of Comp. Sc. Engg} \\
\textit{SRMIST}\\
Chennai, India \\
krsnvijay@gmail.com}
\and
\IEEEauthorblockN{3\textsuperscript{rd} Mettu Kesava Krishna Reddy}
\IEEEauthorblockA{\textit{Dept. of Comp. Sc. Engg} \\
\textit{SRMIST}\\
Chennai, India \\
m.k.k.r.332@gmail.com}
\and
\IEEEauthorblockN{4\textsuperscript{rd} L. Satish Kumar}
\IEEEauthorblockA{\textit{Dept. of Comp. Sc. Engg} \\
\textit{SRMIST}\\
Chennai, India \\
satishaaa111@gmail.com}
}

\maketitle


\begin{abstract}
 The Wikipedia view ,Augmented reality (AR) is a live,
 direct or indirect view of a physical, real-world 
environment whose elements are augmented
 (or supplemented) by computer-generated sensory 
input such as sound, video, graphics or GPS data.
\end{abstract}

\begin{IEEEkeywords}
component, formatting, style, styling, insert
\end{IEEEkeywords}

\section{Introduction}
% TODO:3 para intro %
\section{Literature Survey}
% TODO:Atleast 4 papers %

\section{Proposed Methodology}
% TODO: Use more Images %
\subsection{User Experience}
\subsection{Interaction}
% TODO:HandTracking %
\subsection{Design}

\section{Prototype System}
% TODO: Include Open Source Libraries %
As the project should be easily accessible, We've developed it using Web Technologies so that it can run on any device that has a web browser and is not limited by OS or Hardware.
Since it is a web application no installation is required and can be accessed anywhere.


\subsection{Frameworks, Tools used for development}
\subsubsection{Three.js} Three.js is a cross-browser JavaScript library/API used to create and display animated 3D computer graphics in a web browser. Three.js uses WebGL.
\subsubsection{AR.js}AR.js is efficient Augmented Reality for the web that also works well on mobile devices too. It is all in javascript and runs in any browser with WebGL and WebRTC. 
\subsubsection{A-Frame} A-Frame is a powerful entity-component framework that provides a declarative, extensible, and composable structure to three.js.
A-Frame’s core components include geometries, materials, lights, models, shadows, physics, motion capture and augmented reality.
\subsubsection{Tracking.js} The tracking.js library brings different computer vision algorithms and techniques into the browser environment. By using modern HTML5 specifications real-time color tracking, face detection, motion tracking is possible.
\subsection{Prototype Architecture}

\subsection{Modules}
With the help of advanced augmented-reality technology such as computer vision and object recognition, the information about the surrounding real world of the user becomes interactive and able to be digitally manipulated. 
\subsubsection{Marker Tracking}
Marker based AR uses a Camera and a visual marker to determine the center, orientation and range of its spherical coordinate system. Using a marker makes the application run faster as the object to detect is known before hand.
\subsubsection{Gesture Recognition}
Interaction with digital environment via hand gestures makes the experience quick and natural. Gestures to scale,manipulate or rotate the 3D Scene gives the user a sense of freedom .
\subsubsection{3D Scene Rendering}
Rendering the virtual model using webGL and mapping it on the marker. Visual artifacts like shadows, lighting, texture and distance are added to the final render that is displayed on the screen.
\subsubsection{3D Object Interaction}
The real space is used for data input, users perform gestures that is detected by the camera in real-time and the 3D virtual scene is projected onto the real environment reflecting the changes made by the user.

\subsubsection{Perspective Camera}
This camera mode is designed to mimic the way the human eye sees. It is the most common projection mode used for rendering a 3D scene. When there is camera movement the perspective of the 3D virtual scene is also changed accordingly
\section{Usability Evaluation}
\subsection{Evaluation Result}
% TODO:Table %
\section{Conclusion}

\section*{Acknowledgment}

\section*{References}

\end{document}
