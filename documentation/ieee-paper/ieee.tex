\documentclass[conference]{IEEEtran}
\begin{document}

\title{ARchitect: AR Based Construction Simulator}

\author{\IEEEauthorblockN{1\textsuperscript{st} L. Sai Kiran Reddy}
\IEEEauthorblockA{\textit{Dept. of Comp. Sc. Engg} \\
\textit{SRMIST}\\
Chennai, India \\
saikiranreddy1001@gmail.com}
\and
\IEEEauthorblockN{2\textsuperscript{nd} Vijay Krishna Vallabhaneni}
\IEEEauthorblockA{\textit{Dept. of Comp. Sc. Engg} \\
\textit{SRMIST}\\
Chennai, India \\
krsnvijay@gmail.com}
\and
\IEEEauthorblockN{3\textsuperscript{rd} Mettu Kesava Krishna Reddy}
\IEEEauthorblockA{\textit{Dept. of Comp. Sc. Engg} \\
\textit{SRMIST}\\
Chennai, India \\
m.k.k.r.332@gmail.com}
\and
\IEEEauthorblockN{4\textsuperscript{th} L. Satish Kumar}
\IEEEauthorblockA{\textit{Dept. of Comp. Sc. Engg} \\
\textit{SRMIST}\\
Chennai, India \\
satishaaa111@gmail.com}
}

\maketitle


\begin{abstract}
 Augmented reality (AR) is a growing phenomenon, it is used to enhance the natural environments 
 or situations and offer perceptually enriched experiences.
 With the help of advanced AR technology the information about the surrounding 
 real world of the user becomes interactive and digitally manipulatable.

In this paper, a user-friendly, portable and platform independent AR system 
is proposed that allows users to manipulate/construct 3D virtual buildings on a real table-top using Web-AR so that it is accessible to everyone anywhere. Since it is virtual they are also free to interact with the model in real time by adding or  deleting parts to the building or scaling portions of it to examine in greater detail. Simulated tests like structural integrity, Air Flow and Lighting can be performed on this virtual model. 

This project aims to make 3D modeling of a building easier by providing a
 natural and more intuitive interaction. It makes use of the in-built camera
 to map the digital objects into the real world with the help of markers and
 JavaScript libraries/frameworks like Three.js,AR.js and A-Frame.js.
 
 Augmented reality is a powerful way to bring the physical and digital worlds
 together. AR places digital objects and useful information into the real world  around us, which creates a huge opportunity to make our phones more intuitive, more helpful and a whole lot more fun.
\end{abstract}

\begin{IEEEkeywords}
component, formatting, style, styling, insert
\end{IEEEkeywords}

\section{Introduction}
In the present day there is an increase in the use of AR technology in both commercial and industrial applications, many companies are adopting the new vision towards designing of their products using AR and VR which not only reduces the designing cost but also helps in the testing of many real time scenarios and provides a natural interaction that is easy to use.

Ford is using the AR technology in visualizing the structural design of the new cars and is also using the simulated effects of wind and snow on 
the engine performance ,Space travel revolutionizing company Space-x simulates the effect of gravity and rockets orbit placement using AR, 
The Boeing is using AR to design the new Air Force One in which it simulates the effects of engine propulsion ,and also for the core structure designing which is a heavy in-built metal body to withstand any missile attacks on the plane. Thus simulation plays a major role in designing a product where the imperfections can be detected and rectified in the early stages of development and helps the company to create a final product that is robust and reliable.

This paper proposes an idea to visualize the effect of natural calamities on structures and offer an in-depth analysis of the structural weak points of a design. With AR , Instead of displaying the blueprints of architectural design on a 2-D board or on auto-cad we can visualize the 3-D models in a real time environment where it can be interacted by the user. Currently AR applications focus mainly on the construction and alignment of objects in real world environment, This proposal takes that concept one step further by performing realistic simulations of the natural calamities such as floods,  drought, wind, lighting and effect of rain on the structural and materialistic properties of the simulated designs by using a particle based physics engine. 

At present the effect of natural calamities might some times turn out to be incalculable when estimating human and structural loss. The damage can be partially prevented by designing a near to perfect structure that is able to withstand the calamities in a realistic physics simulation. Thus simulations not only help in estimating the damage but can also help us to prevent the disasters to an extent. In this proposal the simulation of natural phenomenons such as Hurricanes, Earthquakes and Lightning on the materialistic and structural properties of the design can help in detecting the structural weak points of a design.

The alignment and design of AR based models according to real-time scenarios requires precision and correctness. The tools used in designing such effective structural designs and simulations are three.js, AR.js, A-Frame, Tracking.js . The various user interactive formats proposed in the design are Gesture Recognition, Marker Tracking, 3D Object Interaction, 3D Scene Rendering.
\section{Literature Survey}
% TODO:Atleast 4 papers %

\section{Proposed Methodology}
% TODO: Use more Images %
\subsection{User Experience}
\subsection{Interaction}
% TODO:HandTracking %
\subsection{Design}

\section{Prototype System}
% TODO: Include Open Source Libraries %
As the project should be easily accessible, We've developed it using Web Technologies so that it can run on any device that has a web browser and is not limited by OS or Hardware.
Since it is a web application no installation is required and can be accessed anywhere.


\subsection{Frameworks, Tools used for development}
\subsubsection{Three.js} Three.js is a cross-browser JavaScript library/API used to create and display animated 3D computer graphics in a web browser. Three.js uses WebGL.
\subsubsection{AR.js}AR.js is efficient Augmented Reality for the web that also works well on mobile devices too. It is all in javascript and runs in any browser with WebGL and WebRTC. 
\subsubsection{A-Frame} A-Frame is a powerful entity-component framework that provides a declarative, extensible, and composable structure to three.js.
A-Frame’s core components include geometries, materials, lights, models, shadows, physics, motion capture and augmented reality.
\subsubsection{Tracking.js} The tracking.js library brings different computer vision algorithms and techniques into the browser environment. By using modern HTML5 specifications real-time color tracking, face detection, motion tracking is possible.
\subsection{Prototype Architecture}

\subsection{Modules}
With the help of advanced augmented-reality technology such as computer vision and object recognition, the information about the surrounding real world of the user becomes interactive and able to be digitally manipulated. 
\subsubsection{Marker Tracking}
Marker based AR uses a Camera and a visual marker to determine the center, orientation and range of its spherical coordinate system. Using a marker makes the application run faster as the object to detect is known before hand.
\subsubsection{Gesture Recognition}
Interaction with digital environment via hand gestures makes the experience quick and natural. Gestures to scale,manipulate or rotate the 3D Scene gives the user a sense of freedom .
\subsubsection{3D Scene Rendering}
Rendering the virtual model using webGL and mapping it on the marker. Visual artifacts like shadows, lighting, texture and distance are added to the final render that is displayed on the screen.
\subsubsection{3D Object Interaction}
The real space is used for data input, users perform gestures that is detected by the camera in real-time and the 3D virtual scene is projected onto the real environment reflecting the changes made by the user.

\subsubsection{Perspective Camera}
This camera mode is designed to mimic the way the human eye sees. It is the most common projection mode used for rendering a 3D scene. When there is camera movement the perspective of the 3D virtual scene is also changed accordingly
\section{Usability Evaluation}
\subsection{Evaluation Result}
% TODO:Table %
\section{Conclusion}

\section*{Acknowledgment}

\section*{References}

\end{document}
